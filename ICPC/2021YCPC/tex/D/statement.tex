Two villages are connected together by a two-way road. 
In an environment, there are many villages. 
A traveler starts from a village (source) and moves to a predefined village 
(destination). 
To do so, the traveler moves along the roads to visit villages until the 
traveler reaches the destination. 
The traveler can find some moon stones at a village. 
Each road has a travel cost. 
Determine a path for the traveler so that the traveler takes the lowest cost. 
If there are two paths with the same cost, 
the one with the highest amount of moon stones is selected.
The total influence cost of a village, $v$, 
is the sum of the cost of all the villages in the $k$-neighborhood of 
village $v$. The $k$-neighborhood of village $v$ does not include the 
village $v$ itself. 
For two different villages $u$ and $v$, village $u$ is a member of the 
$k$-neighborhood of village $v$ \emph{iff} there are at most $k$ roads that are 
visited for moving from village $v$ to village $u$.

The travel cost from one village $v_s$ to an adjacent village $v_a$ 
via road $r$ is computed as:
$$\mbox{Travel cost} = c_t(v_s) + c_r$$
where road $r$ connects villages $v_s$ and $v_a$, 
$c_t(v_s)$ is the total influence cost of $v_s$, 
and $c_r$ is the cost of the road. 

The villages have unique names. 
A name is formed by four letters in lowercase.
